\nc{awk}
Pattern-directed scanning and processing language.

\begin{itemize}[label={-}, leftmargin=*]
\addtolength{\itemsep}{-2pt}
\item Print every line in file.\\
{\tt \$ awk `{print}' file}

\item Print the line matching the pattern.\\
{\tt \$ awk `/pattern/ {print}' file}

\item Print the 3rd column and the record number.\\
{\tt \$ awk `{print NR,\$3}' file}
\item Print lines with more than 10 characters.\\
{\tt \$ awk `length(\$0) > 10' file}

\item Select a different column-separator.\\
{\tt \$ awk -F: `{print \$1}' <file>}

\item Concatenate several commands.\\
{\tt \$ echo "Hi Tom" | awk `{\$2="Adam"; print \$0}'}

\item Count and print matched pattern.\\
{\tt \$ awk `/a/{++cnt} END {print "Count=", cnt}' <file>}

\item The commands can be stored in a file.\\
{\tt \$ awk -f <command-file> <file>}

\item Pre- and post-process with {\tt awk}.\\
{\tt \$ awk `BEGIN{print "Start"} {print \$0} END{print "End"}' file}

\item You can define you own variables.\\
{\tt \$ awk `myVar="Hello" print myVar} file\\
But some variable names have special meanings.\\
{\tt CONVFMT, FS, NF, NR, NFR, FILENAME, RS, OFS, ORS...}

\item There are control structures (if, for, while...), formatting, user-defined functions...\\
\end{itemize}